% !TeX root = ./thesis.tex

\chapter{chapter name}
{THE MOLECULAR PATHWAYS OF SEPSIS.
}\label{the-molecular-pathways-of-sepsis.}

\section{ The Innate Immune System.
}\label{the-innate-immune-system.}

The host's body initially promotes an innate pro-inflammatory response
as a response to pathogens, which is arbitrated by antigen-presenting
cells (APCs). These cells express pattern-recognition receptors (PRRs)
on their surface, which can detect pathogen-associated molecular
patterns (PAMPs) expressed on a pathogen's surface or through the
release of damage-associated molecular patterns (DAMP) as a result of
tissue damage {[}39{]}. Upon recognition, the PRRs activate various
receptors such as the nucleotide-binding oligomerization domain
(NOD)-like receptors and Toll-like receptors (TLRs). These cause a
cascade of reactions across multiple pathways that promote the
manufacture of pro-inflammatory cytokines and chemokines, which trigger
second messenger cascades, resulting in amplified immune responses
{[}35{]}.

Cytokines and chemokines are crucial mediators of immune responses as
they enable the recruitment of leukocytes to the site of
infection/injury and increase the permeability of the endothelial
vasculature, allowing for the localization of leucocytes {[}36, 37{]}.
Cytokines and chemokines also facilitate communication between immune
cells and their mediators and among adipocytes, fibroblasts, and
endothelial cells. Additionally, cytokines and chemokines allow for
interactions between the various cascade systems responsible for
eliciting immune responses to occur. Given the complexity of the immune
system, this particular intricacy has made it extremely difficult for
researchers to understand the molecular mechanisms of the immune system
{[}38--41{]}

The name interleukin was suggested in 1979, which means "communication
between leucocyte" "{[}42, 43{]}. Many of these proteins are produced by
and act on leukocytes, but cells from other tissues can also secrete
them. They exert complex immune-modulatory functions, including cell
proliferation, maturation, migration, and adhesion {[}44, 45{]}. The
NOD-like receptor group aggregates to form larger protein complexes
called inflammasomes {[}7{]}. These protein complexes play a vital role
in the production and release of critical cytokines IL-1$$\beta$$and IL-18.
They are also involved in the formation of caspases, which are
implicated in apoptosis {[}48{]}. These pro-inflammatory cytokines
elicit leukocyte proliferation, upregulate chemokine expression and
express tissue factor production, and induces the production of hepatic
acute phase reactants, which are important mediators produced in the
liver during times of acute and chronic inflammation {[}46, 49{]}.
During sepsis, these immune responses are amplified, leading to damage
and death of tissues and cells. Recent studies that have analyzed the
association between IL-18 levels and mortality {[}50--52{]} suggest the
role of inflammasomes and autophagy as potential targets in the
treatment of sepsis.

\section{ Toll-Like Receptors }\label{toll-like-receptors}

TLRs are a type of PRR expressed on APCs. These TLRs are thought to play
a very crucial role in the induction of innate immunity. This family of
type I transmembrane receptorsSystemic inflammatory response syndrome (SIRS) is a generalized immunological response against a vast range of pro-inflammatory pathologies, including infection, injury, trauma, and burns. SIRS is often characterized by significant changes in the body temperature and the onset of tachycardia, rapid breathing, and abnormalities in white blood cells (WBCs) and red blood cell (RBCs) counts and can give rise to systematic multi-organ dysfunction. When the onset of infection causes SIRS, the phenomenon is known as sepsis. Sepsis is a life-threatening inflammatory response that can give rise to systematic multi-organ dysfunction and failure caused by either trauma or infection.
Despite modern advances in elucidating the pathophysiology of sepsis, the condition remains one of the primary causes of mortality and morbidity in intensive care units (ICUs) worldwide. Current estimates suggest that sepsis affects more than 30 million people and accounts for more than six million deaths per annum worldwide. Based on the Surviving Sepsis Campaign data from 2012, 41$\%$ and 28.3% of reported deaths from sepsis occurred in Europe and the United States, respectively [1]. The study also found sepsis to be the most expensive health care condition in the United States annually, setting back American hospitals by USD 20 billion in 2011 alone [3]. These financial and mortality costs make the investigation of sepsis's molecular mechanisms a top priority to elucidate possible immune modulation therapies to more effectively treat patients afflicted by sepsis.
Severe sepsis is when the host's reaction to infection causes a systemic cascade of organ failures in a manner referred to as septic shock [1-6]. Sepsis is believed to cause organ failure through the uncontrolled upregulation of systemic immune responses. However, in light of medical and scientific advancements, ICU survival rates have improved, which led to the detection of the immunosuppression phase in the later stages of sepsis pathophysiology, ultimately explaining the high mortality rates. This syndrome was termed "compensatory anti-inflammatory response syndrome" (CARS) by Bone in his 1996 paper [4]. Similar to SIRS, CARS is a complex immune system response to severe infection; however, CARS is believed instead to be a condition marked by systematic inhibition of the immune system that restores homeostasis after the period of extreme inflammation. This led scientists and medical professionals to use the terms SIRS and CARS to differentiate the host's pro- and anti-inflammatory responses to a broad range of infectious and noninfectious stimuli [6–9].
While initial studies categorized CARS as the phase that appears at the end of or even after SIRS, researchers have since found evidence of pathways that support the idea that CARS is not a part of SIRS. Instead, CARS may exist entirely separately from SIRS and encompass an additional set of cellular and molecular interactions and pathogenesis pathways different from those of SIRS. However, CARS may also significantly influence sepsis and lead to adverse outcomes: while earlier studies of the pro-inflammatory phase of sepsis have helped to improve survival rates in the ICU, the emergence of an immunosuppression phase in the later stages of sepsis pathophysiology often left the patient vulnerable to secondary infections, which could explain the high mortality rates [10]. Indeed, later studies revealed that the anti-inflammatory responses elicited by CARS induce a severe immunosuppressed state wherein the immune system cannot recover despite eradicating pathogens from the body, which, as a phenomenon, has been termed “immune paralysis” [5].
Modern advances have reduced the rates of deaths occurring during sepsis' initial stages as homeostasis is reestablished early on in the disease's pathophysiology. However, those patients who fail to achieve homeostasis during the early phases of SIRS/CARS enter a state marked by high mortality and morbidity rates, typically exhibiting severe weakness, malnutrition, chronic infections, and cognitive decline, which has come to be known as chronic critical illness [10–13].
Data from 2009 indicate that the annual health care costs for patients with chronic critical illness exceeded $\$$20 billion. The majority of these patients (> 60$\%$) were admitted with a sepsis diagnosis [12], and only 20% were ultimately discharged home; more than 40% were discharged to long-term acute care and skilled nursing facilities [11, 12], while 30% died in the hospital [12].
Due to its associated high mortality rates, CARS soon became a target for immune-modulating therapies [14]. However, despite extensive preclinical research into possible immunomodulatory therapies for CARS, not many treatment solutions to date have been implemented [15]. Later studies found that, during CARS’ immunosuppressive phase, an increase in the levels of pro-inflammatory cytokines such as C-reactive protein (CRP), interleukin (IL)-6, IL-1Ra, and tumor necrosis factor (TNF) receptor [14, 16] occurred, together with a substantial rise in the recruitment and release of immature myeloid leukocytes associated with chronic inflammation [46]. These studies have supported the design of a more fluidic model of sepsis with simultaneous inflammatory and immunosuppressive processes. This evidence eventually led to the replacement of the traditional SIRS/CARS model with the concept of persistent inflammation–immunosuppression catabolism (PICS) [6].
PICS is characterized by a low but constant, chronic state of inflammation that paralyzes the host's immune system while exerting drastic catabolic effects on the body mass’ nutritional intervention [7, 13]. The key adaptive immune features that once typified CARS are now understood to fall under the larger umbrella of PICS. These processes include immune cell metabolic failure, decreased T-cell numbers, lymphocyte dysfunction, increased apoptosis, increased T-cell suppressor function, reduced T-cell repertoire, significant shifts in cytokine polarization toward humoral and TH2 cytokines, diminished membrane-associated human leukocyte antigen receptors, and epigenetic modifications secondary to the cell microenvironment [1, 22, 23, 28–33]. The definitions and diagnostic criteria for sepsis and PICS are defined in Table 1.1.
While the etiology or pathophysiology of PICS has not been completely elucidated, extensive studies on pro-and anti-inflammatory cytokines and chemokines have revealed the sheer breadth of sepsis and its many modes of action. Recently developed controversial theories suggest that the role of endotoxins and immunosuppressive SIRS medication might be secondary to the role of endogenous molecules like catecholamines [18, 36], corticosteroids [19, 20, 21], and IL-10 [22–27].
 was initially found in drosophila and
were confirmed to possess an extracellular leucine-rich repeat domain
and a highly conserved intracellular Toll/IL-1 receptor (TIR) domain
across plants and animals. This TIR domain enables interactions between
proteins and has been shown to play a vital part in the evolution of
immunity.~

In mammals, scientists have uncovered 10 different kinds of TLRs, with
each one playing a tailored role in innate immunity. These TLRs
recognize highly conserved PAMPs as ligands and have exceptionally low
specificity. In an attempt to understand their intricate workings to
bind with a PAMP/DAMP, their ligand structures have become a topic of
great scientific interest
{[}53--57{]}

%\includegraphics[width=6.34568in,height=3.27860in]{media/image1.png}\textbf{Figure
%2.1} Toll-like Receptors possess an affinity for a diverse range of
%ligands
%
%Source: Medzhitov, R. (2001). Toll-like receptors and innate
%immunity.~Nature Reviews Immunology,~1(2), 135-145.
%
%\url{https://www.nature.com/articles/nri35100529}

Although TLRs' complete workings have not yet been elucidated, recent
data suggest that they often work as dimers. While most form identical
homodimers, some form heterodimers, with each dimer maintaining a unique
affinity for ligands. These PRRs depend upon accessory proteins to aid
in their binding with PAMPs. In particular, TLR4---the most studied
mammalian TLR---plays a crucial role in recognizing the PAMP
lipopolysaccharide (LPS), which is only produced by prokaryotes like
Gram-negative bacteria. TLR4's recognition of~LPS requires MD-2, a small
protein with a currently unknown function that lacks a transmembrane
domain, and CD-14, a high-affinity LPS receptor often expressed on
macrophage surfaces. CD-14 and LPS-binding protein (LBP) present LPS to
MD-2. When these PRRs are activated, they recruit adapter molecules from
the cell's cytoplasm that initiate signaling cascades like MyD88
{[}58--60{]} and Toll-interacting protein TOLLIP.
\newpage