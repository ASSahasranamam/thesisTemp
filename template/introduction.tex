%!TEX root = ../main.tex

\chapter{INTRODUCTION}

\section{Bib\TeX\ Style File}
Systemic inflammatory response syndrome (SIRS) is a generalized
immunological response against a vast range of pro-inflammatory
pathologies, including infection, injury, trauma, and burns. SIRS is
often characterized by significant changes in the body temperature and
the onset of tachycardia, rapid breathing, and abnormalities in white
blood cells (WBCs) and red blood cell (RBCs) counts and can give rise to
systematic multi-organ dysfunction. When the onset of infection causes
SIRS, the phenomenon is known as sepsis. Sepsis is a life-threatening
inflammatory response that can give rise to systematic multi-organ
dysfunction and failure caused by either trauma or infection.

Despite modern advances in elucidating the pathophysiology of sepsis,
the condition remains one of the primary causes of mortality and
morbidity in intensive care units (ICUs) worldwide. Current estimates
suggest that sepsis affects more than 30 million people and accounts for
more than six million deaths per annum worldwide. Based on the Surviving
Sepsis Campaign data from 2012, 41\% and 28.3\% of reported deaths from
sepsis occurred in Europe and the United States, respectively {[}1{]}.
The study also found sepsis to be the most expensive health care
condition in the United States annually, setting back American hospitals
by USD 20 billion in 2011 alone {[}3{]}. These financial and mortality
costs make the investigation of sepsis' molecular mechanisms a top
priority to elucidate possible immune modulation therapies to more
effectively treat patients afflicted by sepsis.

Severe sepsis is when the host's reaction to infection causes a systemic
cascade of organ failures in a manner referred to as septic shock {[}1,
5, 6{]}. Sepsis is believed to cause organ failure through the
uncontrolled upregulation of systemic immune responses. However, in
light of medical and scientific advancements, ICU survival rates have
improved, which led to) the detection of the immunosuppression phase in
the later stages of sepsis pathophysiology, ultimately explaining the
high mortality rates. This syndrome was termed "compensatory
anti-inflammatory response syndrome" (CARS) by Bone in his 1996 paper
{[}4{]}. Similar to SIRS, CARS is a complex immune system response to
severe infection; however, CARS is believed instead to be a condition
marked by systematic inhibition of the immune system that restores
homeostasis after the period of extreme inflammation. This led
scientists and medical professionals to use the terms SIRS and CARS to
differentiate the host's pro- and anti-inflammatory responses to a broad
range of infectious and noninfectious stimuli {[}6--9{]}.

While initial studies categorized CARS as the phase that appears at the
end of or even after SIRS, researchers have since found evidence of
pathways that support the idea that CARS is not a part of SIRS. Instead,
CARS may exist entirely separately from SIRS and encompass an additional
set of cellular and molecular interactions and pathogenesis pathways
different from those of SIRS. However, CARS may also significantly
influence sepsis and lead to adverse outcomes: while earlier studies of
the pro-inflammatory phase of sepsis have helped to improve survival
rates in the ICU, the emergence of an immunosuppression phase in the
later stages of sepsis pathophysiology often left the patient vulnerable
to secondary infections, which could explain the high mortality rates
{[}10{]}. Indeed, later studies revealed that the anti-inflammatory
responses elicited by CARS induce a severe immunosuppressed state
wherein the immune system cannot recover despite eradicating pathogens
from the body, which, as a phenomenon, has been termed ``immune
paralysis'' {[}5{]}.~

Modern advances have reduced the rates of deaths occurring during
sepsis' initial stages as homeostasis is reestablished early on in the
disease's pathophysiology. However, those patients who fail to achieve
homeostasis during the early phases of SIRS/CARS enter a state marked by
high mortality and morbidity rates, typically exhibiting severe
weakness, malnutrition, chronic infections, and cognitive decline, which
has come to be known as chronic critical illness {[}10--13{]}.~

Data from 2009 indicate that the annual health care costs for patients
with chronic critical illness exceeded \$20 billion. The majority of
these patients (\textgreater{} 60\%) were admitted with a sepsis
diagnosis {[}12{]} and only 20\% were ultimately discharged home; more
than 40\% were discharged to long-term acute care and skilled nursing
facilities {[}11,~12{]}, while 30\% died in the hospital {[}12{]}.~

Due to its associated high mortality rates, CARS soon became a target
for immune-modulating therapies {[}14{]}. However, despite extensive
preclinical research into possible immunomodulatory therapies for CARS,
not many treatment solutions to date have been implemented {[}15{]}.
Later studies found that, during CARS' immunosuppressive phase, an
increase in the levels of pro-inflammatory cytokines such as C-reactive
protein (CRP), interleukin (IL)-6, IL-1Ra, and tumor necrosis factor
(TNF) receptor {[}14, 16{]} occurred, together with a substantial rise
in the recruitment and release of immature myeloid leukocytes associated
with chronic inflammation {[}46{]}. These studies have supported the
design of a more fluidic model of sepsis with simultaneous inflammatory
and immunosuppressive processes. This evidence eventually led to the
replacement of the traditional SIRS/CARS model with the concept of
persistent inflammation--immunosuppression catabolism (PICS) {[}6{]}.~

PICS is characterized by a low but constant, chronic state of
inflammation that paralyzes the host's immune system while exerting
drastic catabolic effects on the body mass' nutritional intervention {[}7, 13{]}
%{[}\href{https://www.ncbi.nlm.nih.gov/pmc/articles/PMC7019692/\#B7-jcm-09-00191}{7},
%\href{https://www.ncbi.nlm.nih.gov/pmc/articles/PMC7019692/\#B13-jcm-09-00191}{13}{]}.
The key adaptive immune features that once typified CARS are now
understood to fall under the larger umbrella of PICS. ("(PDF) A rapidly
progressing lymphocyte exhaustion after...") These processes include
immune cell metabolic failure, decreased T-cell numbers, lymphocyte
dysfunction, increased apoptosis, increased T-cell suppressor function,
reduced T-cell repertoire, significant shifts in cytokine polarization
toward humoral and TH2 cytokines, diminished membrane-associated human
leukocyte antigen receptors, and epigenetic modifications secondary to
the cell microenvironment {[}1, 22, 23, 28--33{]}.~The definitions and
diagnostic criteria for sepsis and PICS are defined in Table 1.1.

While the etiology or pathophysiology of PICS has not been completely
elucidated, extensive studies on pro- and anti-inflammatory cytokines
and chemokines have revealed the sheer breadth of sepsis and its many
modes of action. Recently developed controversial theories suggest that
the role of endotoxins and immunosuppressive SIRS medication might be
secondary to the role of endogenous molecules like catecholamines {[}18,
36{]}, corticosteroids {[}19, 20, 21{]}, and IL-10 {[}22--27{]}.
