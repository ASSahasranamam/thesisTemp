\noindent
% Start typing your abstract below this line

Adithya Subramanian Sahasranamam
Sepsis is a complex, life-threatening syndrome that can lead to systemic organ failure and dysfunction. Due to its high morbidity and mortality rates, it has become a critical global health issue. Although many factors are at play during sepsis, the primary ones include abnormal inflammation and a lack of oxygen supply to the tissues and muscles. The interleukin-1 receptor–associated kinase (IRAK) family plays a crucial role in eliciting innate immune responses and switching to adaptive immune responses in the presence of pathogens. IRAKs are vital components in the interleukin-1 receptor signaling pathway and some Toll-like receptor signaling cascades that elicit inflammatory reactions in response to injury and infection. Disturbances in the homeostasis of IRAK signaling cascades can lead to a plethora of immunological problems. This paper seeks to understand the molecular mechanisms of IRAK-1 and compare its splice variants and polymorphisms while considering inflammation and sepsis.
