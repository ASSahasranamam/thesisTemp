\noindent
% Start typing your abstract below this line


% Delete everything below
\noindent
Instead of creating a \verb#README# file on how to use the \LaTeX\ style file
(\verb#thesis.cls#), I will use this page to briefly describe each page
of the dissertation, and read Chapter~1 on using Bib\TeX\ style file and formating your figures and tables. I assume you are familiar with \LaTeX.  This page is the
{\em abstract} page.  To modify the {\em abstract} page, edit
\verb#abstract.tex# file.  The next page is {\em title}  page. You need modify this page if your major is not mathematics. Edit \verb#thesis.cls# file and modify command \verb#\maketitle# which is from lines 376 to 398. {\em
Copyright} page is next; no modification is required.  Next is the {\em approval} page.  I illustrate the
general format followed by two examples.  Note that the title of your
advisor and committee members have to match with NJIT's graduate catalogue.
If your committee member(s) is not a professor at NJIT, you may have to ask
him/her for the correct title.  To fill out your committee's information,
edit \verb#approval.tex# file.  In case, there are more than five people on
your committee, you may have to adjust the space so they fit on one page
nicely.  To adjust the gaps, edit the file \verb#thesis.cls# and adjust the values
on lines 490.  The {\em biography} page is next. This is where you
provide some of your personal information such as birthday, birthplace and
background education, etc.  If you have published or presented any
seminars, list them here. To modify your {\em biography} page, edit
\verb#biography.tex# file.  The file should be easy to understand.  For
education background (lines 26--33) and publications (lines 38--58), I
show the general format followed examples. Don't forget to list the
most recent item first. If you don't have any publication or presentation,
you can remove or comment out lines 38--58 and lines 544 and 546 in \verb#thesis.cls# file. Note that the format of publications mirrors the default bibliography format. Therefore, if you use other bibliography format, you should mirror your publications format with it for consistency. Also, the references to line
numbers are only correct before you make any modification to
\verb#biography.tex# file.  The next two pages are {\em dedication} and {\em
acknowledgment} pages.  On the {\em dedication} page, you can write pretty
much anything that you think appropriate such as citing a poem, song lyrics, quoting someone or even including a picture of your special someone.  And you can do all that by editing the
\verb#dedication.tex# file.  You may want to put your dedication in the
middle of the page.  You can do that by adjust the value on line 637 in
\verb#thesis.cls# file.  Finally, the content of \verb#acknowledgment.tex#
file is for the {\em acknowledgment} page.  This is where you tell everyone
how much you love your advisor and thank all the people who has helped you.
Don't forget to thank the organization(s) that funded your research.  The
``real'' thesis starts after the {\em acknowledgment} page, and it goes after
line 62 in \verb#thesis.tex# file.\footnote{[My lawyer advises me to include this] \\
{\sc Disclaimer}:  The names, events,
and publications are fictional.  They are provided as the sole purpose of
clarifying the general format.  Any resemblance to real names, events and
publications is entirely accidental. \\
{\sc Use of Style file}: The style file is provided ``as is''. In no event shall I or previous contributors be liable for any damage causes by the use of style file such as the style file ``eats'' your dissertation which in itself is an excellent lesson on backup.}
